\begin{lec}[2012-01-19]\end{lec}
\stepcounter{section}

\begin{thm}\label{thm23.1}
The incidence matrix $M$ of digraph $D$ is totally unimodular.
\end{thm}
\begin{proof}
Let $B$ a square $t \times t$ submatrix of $M$. We prove $\det(B) \in
\{-1,0,1\}$ by induction on $t$.

For $t = 1$, the result is trivial.

For $t > 1$, we distinguish three cases:

\begin{enumerate}
\item[Case $1$:] $B$ has a column with only zeros. Then $\det(B) = 0$.
\item[Case $2$:] $B$ has a column with exactly one non-zero. By Laplacian
expansion, it follows that $\det(B) = \pm \det(B')$ where $B'$ is the $t-1
\times t-1$ submatrix after deleting the row and column containing the
non-zero.

By induction, $\det(B') \in \{ -1,0,1 \}$ and hence $\det(B)$ as well.
\item[Case $3$:] Every column of $B$ contains two non-zeros, exactly one $1$
and one $-1$. If we take the linear combination of all rows with $\lambda_i
= 1, i = 1, \dots, t$, the result is a zero vector, hence the rows are
linear depending and thus $\det(B) = 0$.
\end{enumerate}
\end{proof}

\begin{cor}\label{cor23.2}
The shortest path problem, the max flow problem and the min cost flow
problem have, written as LP, only integer vertices.
(If the capacities of the arcs are integer).
\end{cor}
\begin{proof}
Follows from Theorem~\ref{thm23.1} and Corollary~\ref{cor22.4}.
\end{proof}

\tocsubsection{LP Duality}
\begin{thm}\label{thm23.3}
Let $A$ be a $m \times n$ matrix, $b \in \R^m$ and $c \in \R^n$. Further,
let $P = \{ x \mid A x \leq b \}$ and $Q = \{ x \mid A^T y = c, y \geq 0 \}$.

\begin{enumerate}[(i)]
\item \new{weak duality}{schwache Dualität} \\
If $x \in P$ and $y \in Q$, then $c^T x \leq b^T y$
\item \new{strong duality}{starke Dualität} \\
$\underbrace{\max \{ c^T x \mid Ax \leq b \}}_{\text{primal LP}} =
\underbrace{\min \{ b^T y \mid A^T y = c, y \geq 0 \}}_{\text{dual LP}}$ \\
if both $P$ and $Q$ are non-empty
\end{enumerate}
\end{thm}
\begin{proof}
Optimierung A, exercise sheet.
\end{proof}

\tocsubsection{Total Dual Integrality (TDI)}
\begin{defn}\label{def23.4}
A rational system $A x \leq b$ is \new{total dual integral}{total dual
integral} if for every integer vector $c$ such that $z_{LP} = \max \{ c^Tx \mid Ax
\leq b\}$ exists (finite), the dual $\min \{ b^Ty \mid A^Ty=c, y \geq 0\}$
has an integer optimal solution.
\end{defn}

\begin{xmp+}
Let $b_1, b_2 \in \Z$, $c_1, c_2 \in \Z$ and consider the primal problem
\begin{align*}
&\max & c_1 x_1 &+ c_2 x_2 \\
&s.t. & x_1 &+ x_2 & \leq b_1 \\
&& 2 x_1 &+ x_2 &\leq b_2
\end{align*}
The dual problem is
\begin{align*}
&\min & b_1 y_1 &+ b_2 y_2 \\
&s.t. & y_1 &+ 2y_2 & = c_1 \\
&& y_1 &+ y_2 & = c_2 \\
& y_1 \geq 0, y_2 \geq 0
\end{align*}
Hence, $y_1 = 2 c_2 - c_1$, $y_2 = c_1 - c_2$.

If $c_2 > c_1$, the primal problem is unbounded ($x_1 \rightarrow -\infty$,
$x_2 \rightarrow \infty$).
\end{xmp+}