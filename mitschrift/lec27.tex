\begin{lec}[2011-02-02]\end{lec}
\stepcounter{section}

\tocsubsection{Matroids and Independencs Systems}

\begin{qstn}
The minimum spanning tree problem can be solved by the greedy algorithm:
always take as next edge a minimum weight one that does not close a cycle.
Are there other problems, where the greedy algorithm solves the problem to
optimality?

General Independence Systems.
Let $E$ be a finite set. $2^E$ denotes the set of all subsets of $E$.
\end{qstn}

\begin{defn} % 27.1
A set $\mathcal{I}\subseteq 2^E$ is an \new{independence
system}{Unabhängigkeitssystem} (IS) on $E$, if $\mathcal{I}$ satisfies the following
conditions:
\begin{itemize}
\item[(I.1)] $\emptyset \in \mathcal{I}$
\item[(I.2)] $F \subseteq G \in \mathcal{I}\Rightarrow F \in \mathcal{I}$
\end{itemize}
Often, we call the pair $(E, \mathcal{I})$ an IS. The subsets of $E$ contained in
$\mathcal{I}$ are \new{independent sets}{unabhängige Mengen}; all other subsets of
$E$ are \new{dependent sets}{abhängige Mengen}. The inclusion-wise minimal
dependent sets of $E$ are the \new{circuits}{} (or \new{cycles}{}), i.e. $C
\subseteq E$ is a circuit if $C$ is dependent and $C \setminus \{i\}$ is
independent for all $i \in C$.

The set of all circuits is called the \new{circuit system}{} and denoted by
$\mathcal{C}$. Given $F \subseteq E$, every independent subset of $F$ which
is not contained in another independent subset of $F$ is called a
\new{basis}{Basis} of $F$, i.e.
\[
	B \text{ basis of } F \Leftrightarrow (B, B' \in \mathcal{I}, B \subseteq B'
	\subseteq F \Rightarrow B = B')
\]
The set of all bases of the ground set $E$ is called the \new{basis
system}{Basissystem} (regarding $\mathcal{I}$) and is denoted by $\mathcal{B}$.

For every set $F \subseteq E$ is the \new{rank}{Rang} of $F$ defined by
\[
	r(F) := \max \{ |B| \mid B \text{ basis of } F \}
\]
\end{defn}

\begin{xmp+}
Let $\mathcal{I}$ be the set of all forests of $E$.
\begin{itemize}
\item Circuits: cycles in $G = (V, E)$
\item basis of $F = E$: spanning tree (if $G$ is connected)
\item basis of $F \subset E$: spanning trees oin the connected componens of
$(V,E) \rightarrow$ maximum weighted spanning tree is a special case of
Definition~\ref{def:dfn27.3}. All bases are spanning trees and thus have
$|V|-1$ edges per component $\Rightarrow$ (27.4) is satisfied.
\item $r(E) = n-1$ if $G$ is connected
\item $r(F) = n- \text{\#components of } (V(F), F)$
\end{itemize}
\end{xmp+}

\begin{lem}\label{lem:lem27.2}
For an independent set $F \subseteq E$, it holds that
\begin{enumerate}[(a)]
\item $F$ is a basis
\item $r(F) = |F|$
\end{enumerate}
for a circuit $F \subseteq E$ it holds that
\begin{enumerate}[(a)]
\item $F \setminus \{i\}$ is a basis of $F$ for all $i \in F$
\item $r(F) = |F| - 1$
\end{enumerate}
In general it holds, that 
\begin{enumerate}[(a)]
\item $r(F) \leq |F|$ (subcardinality)
\item $F \subseteq G \Rightarrow r(F) \leq r(G)$ (monotonicity)
\item $F, F_i$ $(i \in K)$ with $F_i \subseteq F$ and $| \{ i \in K \mid e
\in F_i \}| = k \;\forall e \in F$.
It holds
\[
	k \cdot r(f) \leq \sum_{i \in K} r(F_i) \text{ (strong subadditivity)}
\]
\end{enumerate}
No proof.
\end{lem}

\begin{defn}\label{def:defn27.3}
Let $c : E \rightarrow \R$ be a weight function on $E$. For $F \subseteq E$
let $c(F) := \sum_{e \in F} c(e)$. The optimization problem on the IS $\I
\subseteq 2^E$ is $\max \{ c(I) \mid I \in \mathcal{I}\}$.
\end{defn}

\begin{xmp+}
$G = (V,E)$. A subset $S \subseteq V$ is stable (clique) if every pair in
$S$ is non-adjacent. The set of stable sets (cliques) is an IS on $V$. The
maximum weighted stable set problem (clique) can be described as an IS
optimization problem. Note those two problems are \npoly-complete.
\end{xmp+}

\begin{defn}\label{def:defn27.4}
A \new{matroid}{} $\mathcal{M}$ consists of a ground set $F$ and an IS $\I
\subseteq 2^E$ witch satisfies the following equivalent conditions:
\begin{itemize}
\item[(I.3)] $I,J \in \mathcal{I}, |I| = |J|-1 \Rightarrow \exists j \in J\setminus
I$ with $I \cup \{j\} \in \mathcal{I}$
\item[(I.3')] $I,J \in \mathcal{I}, |I| < |J| \Rightarrow \exists K \subseteq J
\setminus I$ with $|I \cup K| = |J|$ such that $I \cup K \in \mathcal{I}$
\item[(I.3'')] $F \subseteq E$ and $B, B'$ bases of $F \Rightarrow |B| =
|B'|$
\end{itemize}
\end{defn}

Matching is not matroid.
A matroid defined on a graph is called a \new{graphical matroid}{}

\begin{lstlisting}[caption=Greedy algorithm for IS]
1	Sort the weights in non-increasing order
2	Set $I_g := \emptyset$
3	FOR $i = 1$ TO $n$ DO
		IF $c_i \leq 0$ THEN GOTO 4
		IF $I_g \cup \{i\}$ is independent THEN set $I_g := I_g \cup \{i\}$
	ENDFOR
4	RETURN $I_g$
\end{lstlisting}

To analyze, let $F \subseteq E$ and define the \new{lower rank
function}{untere Rangfunktion} of $F$ as
\[
	r_L(f) := \min \{|B| \mid B \text{ basis of } F\}
\]
Further, let 
\[
	q := F \subseteq E, \min_{r(g) > 0} \frac{r_L(F)}{f(F)}
\]
be the \new{rank quotient}{Rangquotient}.

\begin{thm}\label{thm:thm27.5}
Let $\mathcal{I}$ be an IS on $E$. Let $I_g$ be the greedy solution and let $I_{OPT}$
an optimal solution of $\max \{ c(I) \mid I \in \mathcal{I}\}$. Then $q \leq
\frac{c(I_g)}{c(I_{OPT})} \leq 1$ and for every IS there exists weights
$c(e) \in \{0, 1\} \;\forall e \in E$ such that the left inequality is
satisfied with equality.
\end{thm}
No proof.

\begin{cor}\label{cor:cor27.6}
An IS $\mathcal{I}$ on $E$ is a matroid if and only if for all weight functions $C: E
\rightarrow \R$ or $\{0,1\}$ the greedy algorithm provides an optimal
solution.
\end{cor}