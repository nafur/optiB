\documentclass[a4paper,twoside,10pt]{article}

\usepackage[utf8]{inputenc}
\usepackage[T1]{fontenc}
\usepackage{cmbright}
\usepackage{tikz}
\usepackage{amsmath}
\usepackage{enumerate}
\usepackage{amssymb}
\usepackage{amsthm}
\usepackage[normalem]{ulem}
\usepackage{listings}

\definecolor{highlight}{cmyk}{0.8,0,0.38,0.34}
\newcommand{\new}[1]{\emph{\textcolor{highlight}{#1}}}

\newtheoremstyle{style}
   {}
   {}
   {}
   {}
   {\normalfont\bfseries}
   {:}
   {\newline} 
   {}

\theoremstyle{style}

\newtheorem{thm}[subsection]{Theorem}
\newtheorem{sat}[subsection]{Satz}
\newtheorem{defn}[subsection]{Definition}
\newtheorem{defn+}[subsubsection]{Definition}
\newtheorem{lem}[subsection]{Lemma}
\newtheorem{cor}[subsection]{Corollary}
\newtheorem{bsp}[subsection]{Beispiel}
\newtheorem{xmp}[subsection]{Example}
\newtheorem{xmp+}[subsubsection]{Example}
\newtheorem{bem}[subsection]{Bemerkung}
\newtheorem{bem+}[subsubsection]{Bemerkung} %für nicht im krieg skript vorhandene Bemerkungen
\newcounter{question}
\newtheorem{qstn}[question]{Question}
\newcounter{remark}
\newtheorem{rem}[remark]{Remark}

\newcounter{lecture}
\newtheorem{lec}[lecture]{Lecture}
\newtheorem{vl}[lecture]{Vorlesung}

\renewcommand{\labelenumi}{(\alph{enumi})} %erste Ebene (a)
\newcommand{\N}[0]{\mathbb{N}}
\newcommand{\Z}[0]{\mathbb{Z}}
\newcommand{\Q}[0]{\mathbb{Q}}
\newcommand{\R}[0]{\mathbb{R}}
\newcommand{\C}[0]{\mathbb{C}}

\newcommand{\plainset}[1]{\left\{#1\right\}}
\newcommand{\ouptoset}[1]{\plainset{1, …, #1}} %one up to set
\newcommand{\zuptoset}[1]{\plainset{0, …, #1}} %zero up to set
\newcommand{\condset}[2]{\left\{ #1: \; #2\right\}} %condition set

\renewcommand{\Re}[0]{\operatorname{Re}}
\renewcommand{\Im}[0]{\operatorname{Im}}
\newcommand{\sgn}[0]{\operatorname{Sgn}}
\newcommand{\argmin}[0]{\operatorname{Argmin}}
\newcommand{\argmax}[0]{\operatorname{Argmax}}
%\newcommand{\deg}[0]{\operatorname{deg}}
%\newcommand{\det}[0]{\operatorname{det}}

\renewcommand{\thesubsubsection}[0]{\thesubsection(+\arabic{subsubsection})} %Subsections richtig numerieren

\lstset{numbers=left, basicstyle=\ttfamily, numberstyle=\tiny, mathescape=true} %listing style für code

\setlength{\parindent}{0pt}
\setlength{\parskip}{0.25em}

\begin{document}


\setcounter{lecture}{1}
\setcounter{section}{2}

\lec 2011-10-13


\begin{defn}[connected]
A graph is called \new{connected} (zusammenhängend) if there exists a [s,t]-Path between all pairs of vertices $s,t \in V$.
\end{defn}

\begin{defn}[forrest, tree, spanning, forest problem, minimum spanning tree]
A \new{forest} (Wald) is a graph that does not contain a cycle (Kreis). A connected forest is called a \new{tree} (Baum). A tree in a graph (as subgraph) is called \new{spanning} (aufspannend), if it contains all vertices.

Given a graph $G=(V,E)$ with edge weights $c_e \in \mathbb{R}$ for all $e \in E$, the task to find a forest $W \subset E$ such that $c(W):=\sum\limits_{e\in W} $ is maximal, is called the \new{Maximum Forest Problem} (Problem des maximalen Waldes). 
The task to find a tree $T\subset E$ which spans $G$ and which weight $c(T)$ is minimal, is called the \new{Minimum Spanning Tree} (MST) problem (minimaler Spannbaum).
\end{defn}

\begin{lem}
A tree $G=(V,E)$ with at leat 2 vertices has at least 2 vertices of degree 1.
\end{lem}
\begin{proof}
Let $v$ be arbitrary. Since $G$ is connected, $deg(v) \geq 1$. Assume $deg(v)=1$. So $\delta(v)=\{vw\}$. If $deg(w)=1$, we found two vertices with $degree$ 1. If $deg(w)>1$, there exist a neighbour of $w$, different from $v:u$. Now, again $u$ has $degree$ 1 or higher. If we repeat this procedure we either find a vertix of degree 1 or find again \new{new} vertices. Hence, after at most $n-1$ vertices we end up at a vertex of degree 1. 
Now, if $deg(v) \geq 2$, we do the same and find a vertex of degree 1, say $w$. Then repeat the above, staring from $w$ to find a second vertex of degree 1.
\end{proof}

\begin{cor}
A tree $G=(V,E)$ with maximum degree $\Delta$ has at least $\Delta$ vertices of degree 1.
\end{cor}

\begin{lem}
	\begin{enumerate}
	\item For every graph $G=(V,E)$ it holds that 2$|E|=\sum\limits_{u \in V} deg(u)$
	\item for every tree $G=(v,E)$ it holds that $|E|=|V|-1$.
	\end{enumerate}
	\end{lem}

\begin{proof}



	\begin{enumerate}
		\item trivial
		\item Proof by induction. Clearly, if $|V|=1$ or $|V|=2$ it holds. Assumption: true for $n \geq 2.$
		Let $G$ be a tree with $n+1$ vertices. By Lemma 2.3, there exists a vertex $v \in G$ with $deg(v)=1. G-v=G[V\setminus \{v\}$ is a tree again with $n$ vertices and thus $|E(G-v)|=V(G-v)|-1$. Since G differs by one vertex and one edge from $G-v$, the claim holds got $G$ as well.
	\end{enumerate}
\end{proof}


\begin{lem}
If $G=(V,E)$ whith $|V| \geq 2$ has $|E|< |V|-1, G$ is not connected.
\end{lem}

\section*{Algorithm MST}
$min_{x \in X} = -max_{x \in X} -c(x)$ \sout{maximal forest}\\
X spanning trees\\
$min_{x \in X} + (n-1)D= -max_{x \in X} -c(x) (n-1)D =max_{x\in X} \sum \underbrace{D-C_{ij}x_{ij}}_{\geq 0 if D \geq max_{ij \in E} c_{ij}}$

\begin{thm}
Kruskal's Algorithm returns the optimal solution.
\end{thm}
\begin{proof}
Let $T$ be Kruskal's tree and assume there exists a tree $T'$ with $c(T') < c(T)$. Then there exist an edge $e' \in T'\setminus T$. Then $T \cup \{e'\} $ contains a cycle $\{e_1, e_2, \hdots, e_k, e'\}$. Let $ c_f=max_{i=1, \hdots k}c_{l_i} $. 
At the moment Kruskal chooses edge $f$, edge $e'$ cannot be added yet and therefore $c(e')\geq c(f)$. Now exchange $e'$ by $f$ in $T'$. Hence the number of differences beetween $T'$ and $T$ is reduced by one, $C(T'_{new})\leq c(T') < c(T)$. Repeating the procedure results in $c(T) \leq \hdots < c(T)$, a contradiction.
\end{proof}

\include{vl03}
\begin{lec}[2011-10-20]\end{lec}

\begin{thm} % 3.4
Bellman's Algorithm is correct and runs in $O(m + n) = O(n)$.
\end{thm}
\begin{proof}
of runtime:

\[ D(i) = \min\limits_{(i,j) \in A} D(j) + D(i, j) \]

$\Rightarrow$ Every arc is considered once, and thus overall $O(m)$
computations are needed. Initialization costs $O(n)$.
\end{proof}

\begin{bem+}
The running time does not contain the time to find the permutation.
\end{bem+}

Observation 1: We not only found the shortest path from $1$ to $n$, but
also from $i$ to $n$, $i = 2, ..., n$.

Observation 2: We can use a similar procedure for the shortest path from
$1$ to $i$, $i = 2, ..., n$. (with $PREV(i)$ for previous instead of
$NEXT(i)$).

\begin{qstn}
Can we find a shortest path from $1$ to $i$ in a digraph that is
not acyclic, i.e. it contains cycles?
\end{qstn}

\begin{thm} % 4.1
The Moore-Bellman-Algorithm returns the shortest paths from $1$ to $i = 1,
..., n$ provided $D$ does not contain negative-weighted directed cycles.
\end{thm}
\begin{proof}
We call an arc $(i, j) \in A$ an \emph{upgoing} arc (Aufwärtsbogen) if $i < j$ and a
\emph{downgoing} arc (Abwärtsbogen) if $i > j$.

A shortest path from $1$ to $i$ contains at most $n-1$ arcs. If an upgoing
arc is followed by a downgoing arc (or vice versa), we have a \emph{change
of direction} (Richtungswechsel). With at most $n-1$ arcs, at most $n-2$
changes of direction are possible.

Let $D(i, m)$ be the value of $D(i)$ at the end of the $m$-th iteration.
We will show (and this is enough):
\[ D(i, m) = min \{ c(W): \textrm{$W$ is the directed $[1,i]$-path with at most $m$
changes of directions} \} \]
\end{proof}

\include{vl05}
\begin{lec}[2011-10-26]\end{lec}

\setcounter{section}{6}
\setcounter{subsection}{0}

\subsubsection*{Knapsack problem} 

\begin{defn} %6.1
	The \new{Knapsack problem} is defined by a set of items $N = \{ 1, ... , n \}$ weights $a_i \in \N$, value $c_i \in \N$, and a bound $b\in \N$. We search for a subset $S\subset \N$ such that 
	\[
		a(S) = \sum_{i \in S} a_i \leq b \; \text{and} \; c(S) = \sum_{i\in S} c_i \; \text{maximum}
	\]
\end{defn}

Appreach 1: Greedy algorithm

Idea: Items with small weight but high value are the most atrractive ones.

Procedure:
\begin{enumerate}
	\item Sort the items such that $\frac{c_1}{a_1} \leq \frac{c_2}{a_2} \leq ... \leq \frac{c_n}{a_n}$.
	
	Set $S = \emptyset$.
	\item For $i = 1$ to $n$ do
		if $( a(s) + a_i \leq b) $ then
		
			$S = S \cup \{ i \}$
			
		endif
	
	endfor
	\item return $S$ and $c(S)$
\end{enumerate}

\begin{thm}
	The greedy algorithm does \emph{ not } guarantee an optimal solution.
\end{thm}

\begin{proof}
	Let $b=10$, $n = 6$
	
	\begin{align*}
		\begin{matrix}
			i   & 2 & 3 & 4 & 5 & 6 \\
			a_i & 9 & 2 & 2 & 2 & 2 \\
			c_i &19 & 4 & 4 & 4 & 4
		\end{matrix}
	\end{align*}
	
	Greedy: $S=\{1\}$, $c(s)=20$
	
	Optimal: $S=\{2,3,4,5,6,\}$, $c(S)=20$
\end{proof}

Approach 2: Integer Linear Programming

The set of solutions $X$ of a combinatorial optimization problem can (almost always) be written as the intersection of integer points in $\N_0^n$ and a polyhedron $\{x \in \R^n: Ax\leq b\}$

Let $x \in \{0,1\}^n$ be a vector representing all solutions of the knapsack problem:
\[
	x_i = \begin{cases}
		1 \qquad \text{if} i \in S \\
		0 \qquad \text{otherwise}
	\end{cases}
\]

$X = \{0,1\} \cap \{ x \in \R^n: \sum\limits_{i = 0}^n a_ix_i\leq b\}$

Knapsack: $\max \sum{i = 0}^n c_i x_i$

The \new{linear relaxation} (Lineare Relaxierung) of an ILP is the linear program optained by relaxing the integrality of the variables:

$\max \sum_{i=1}^n c_i x_i$

s. t. $\sum_{i=1}^n a_i x_i \leq b, 0 \leq x_i \leq 1 \qquad \forall i \in \{1, ..., n\}$


\end{document}
