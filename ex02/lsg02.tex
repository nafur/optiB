\documentclass[a4paper]{article}
\usepackage[utf8]{inputenc}
\usepackage{fancyhdr}
\usepackage{amsmath}
\usepackage[ngerman]{babel}
\usepackage{amsthm}
\usepackage{tikz}
\pagestyle{fancy}

\usetikzlibrary{automata, positioning, arrows}

\setlength{\parindent}{0pt}
\setlength{\parskip}{1ex}

% Kopfzeile
\lhead{Optimierung B}
\chead{Übung 2}
\rhead{Niklas Fischer 298418 \\ Gereon Kremer 288911}

% Fußzeile
\lfoot{}
\cfoot{Seite \thepage{}}
\rfoot{}

\renewcommand{\thesection}{}
\renewcommand{\thesubsection}{(\alph{subsection})}

\begin{document}

\section{Aufgabe 1}
Lemma: Gegeben ist ein Graph $G_c = (V, E)$ mit Kosten $c(E)$.
Für Mengen $A, B$ mit transitiven, totalen und antisymmetrischen Ordnungsrelationen
$<_A, <_B$ und eine Funktion $f:
A \rightarrow B$ gilt die folgende Äquivalenz:
\[
	\text{$f$ erhält die Relationen auf $A, B$} \Leftrightarrow \text{ein
	MST von $G_c$ ist auch MST von $G_{f(c)}$}
\]

\begin{proof}
Richtung $\Rightarrow$:

$f$ erhält die Ordnung auf den Kantengewichten. Insbesondere ist somit die
kürzeste Kante aus $G_c$ identisch mit der kürzesten Kante aus $G_{f(c)}$. Wenn
der Algorithmus von Kruskal daher $e \in E$ wählt, kann er auf dem
geänderten Graphen sie selbe Kante $e$ wählen. Iterativ kann der Algorithmus
so den selben MST aufbauen. Da der Algorithmus von Kruskal korrekt ist, ist
der auf $G_{f(c)}$ berechnete Spannbaum identisch zum MST auf $G_c$ und
minimal, also ein MST.

Richtung $\Leftarrow$:

Zu zeigen: erhält $f$ die Ordnung nicht, so existiert ein Graph $H_c$, für
den ein MST existiert, der auf $H_{f(c)}$ kein MST ist.

Sei also $f$ eine Funktion, die die Ordnung nicht erhält. Es existieren also
zwei Elemente $a, b$, so dass $a <_A b$, aber $f(b) <_B f(a)$.
Betrachte nun den folgenden Graphen:
\[
\begin{tikzpicture}[auto, node distance=2cm]
	\node (v1) {$v_1$};
	\node (v2) [right of=v1] {$v_2$};
	\node (v3) [right of=v2] {$v_3$};
	\path
		(v1) edge node {$a$} (v2)
		(v2) edge node {$a$} (v3)
		(v1) edge [bend right] node [anchor=base, below] {$b$} (v3)
	;
\end{tikzpicture}
\]
Auf $H_c$ ist ein MST mit Gewicht $2 \cdot a$ durch die Kanten $\{ (v_1, v_2), (v_2, v_3) \}$ gegeben.
Auf $H_{f(c)}$ ist dieser Spannbaum mit Gewicht $2 \cdot f(a)$ jedoch kein MST, da der durch die Kanten $\{ (v_1,
v_2), (v_1, v_3) \}$ gegebene Baum mit Gewicht $f(a) + f(b) <_Y 2 \cdot
f(a)$ bereits ein kleineres Gesamtgewicht hat.
\end{proof}

Die Aufgabe $(a)$ folgt als Korollar aus diesem Lemma.

\subsection{}

Die Funktion $f: x \mapsto x^2$ ist auf den positiven, reellen Zahlen streng monoton und
erhält somit die Ordnungsrelation $<$. Die Aussage folgt somit aus obigem
Lemma.

\subsection{}

Das folgende Beispiel mit $X = \{v_2\}$ widerlegt die Behauptung.
\[
\begin{tikzpicture}[auto, node distance=2cm]
	\node (v1) {$v_1$};
	\node (v2) [right of=v1] {$v_2$};
	\node (v3) [right of=v2] {$v_3$};
	\path
		(v1) edge node {$2$} (v2)
		(v2) edge node {$2$} (v3)
		(v1) edge [bend right] node [anchor=base, below] {$3$} (v3)
	;
\end{tikzpicture}
\]
Der MST $\{ (v_1, v_2), (v_2, v_3)\}$ ist auf dem geänderten Graphen kein
MST.

\subsection{}
Das folgende Beispiel mit $X = \{v_1, v_2\}$ widerlegt die Behauptung.
\[
\begin{tikzpicture}[auto, node distance=2cm]
	\node (v1) {$v_1$};
	\node (v2) [right of=v1] {$v_2$};
	\node (v3) [right of=v2] {$v_3$};
	\path
		(v1) edge node {$1$} (v2)
		(v2) edge node {$2$} (v3)
		(v1) edge [bend right] node [anchor=base, below] {$2$} (v3)
	;
\end{tikzpicture}
\]
Der MST $\{ (v_1, v_2), (v_2, v_3)\}$ ist auf dem geänderten Graphen kein
MST.


\section{Aufgabe 2}

\begin{tabular}{l|l|l|l|l|l|l|l|l}
	& $x_1$ & $x_2$ & $x_3$ & $x_4$ & $x_5$ & $x_6$ & $x_7$ & $x_8$ \\
\hline
init	& $0$ & $\infty$ & $\infty$ & $\infty$ & $\infty$ & $\infty$ & $\infty$ & $\infty$ \\
\hline
$x_1 \xrightarrow{1} x_8$	& $0$ & $\infty$ & $\infty$ & $\infty$ & $\infty$ & $\infty$ & $\infty$ & $1$ \\
\hline
$x_1 \xrightarrow{2} x_2$	& $0$ & $2$ & $\infty$ & $\infty$ & $\infty$ & $\infty$ & $\infty$ & $1$ \\
\hline
$x_1 \xrightarrow{2} x_7$	& $0$ & $2$ & $\infty$ & $\infty$ & $\infty$ & $\infty$ & $2$ & $1$ \\
\hline
$x_1 \xrightarrow{3} x_5$	& $0$ & $2$ & $\infty$ & $\infty$ & $3$ & $\infty$ & $2$ & $1$ \\
\hline
$x_2 \xrightarrow{1} x_3$	& $0$ & $2$ & $3$ & $\infty$ & $3$ & $\infty$ & $2$ & $1$ \\
\hline
$x_8 \xrightarrow{2} x_6$	& $0$ & $2$ & $3$ & $\infty$ & $3$ & $3$ & $2$ & $1$ \\
\hline
$x_8 \xrightarrow{4} x_4$	& $0$ & $2$ & $3$ & $5$ & $3$ & $3$ & $2$ & $1$ \\
\hline
\end{tabular}

\section{Aufgabe 3}

\section{Aufgabe 4}

\subsection{}
Der Algorithmus berechnet kürzeste Wege bezüglich $c'$, da Dijktras
Algorithmus mit den Gewichten $c'$ ausgeführt wird.

Behauptung: Hat ein Weg von $s$ nach $t$ über $c$ eine Gesamtlänge von $w$,
so hat er über $c'$ ebenfalls die Gesamtlänge von $w'$.

Beweis: Die Gesamtlänge über $c$ berechnet sich durch $w = c(s, v_1) +
c(v_1, v_2) + ... + c(v_n, t)$. Über $c'$ ergibt sich $w' = c(s, v_1) +
d(v_1, t) - d(s, t) + c(v_1, v_2) + d(v_2, t) - d(v_1, t) + ... + c(v_n, s)
+ d(t, t) - d(v_n, t)$. Da sich die Summanden $d(v_i, t)$ (ähnlich einer
Teleskopsumme) auslöschen, verbleibt $w' = c(s, v_1) - d(s, t) + c(v_1, v_2)
+ ... + c(v_n, s) + d(t, t) = c(s, v_1) + c(v_1, v_2) + ... + c(v_n, t) -
d(s, t) = w$.

Die Gesamtlänge der Pfade ändert sich also nur um die konstante Länge $d(s,
t)$, Dijkstra liefert also (analog zum Lemma aus Aufgabe 1) über
$c'$ als kürzesten Pfad einen Pfad, der so lang ist wie der kürzeste über
$c$. Der Algorithmus berechnet daher auch bezüglich $c$ kürzeste Wege.

Der angepasste Algorithmus bevorzugt durch die veränderten Gewichte Kanten,
die in Richtung des Ziels $t$ führen. Dies entspricht im wesentlichen der
Idee von $A^*$, dadurch soll der kürzeste Weg schneller gefunden und somit
weniger Aktualisierungen der Entfernungen durchgeführt werden.

\subsection{}

\end{document}
