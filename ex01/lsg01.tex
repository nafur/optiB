\documentclass[a4paper]{article}
\usepackage[utf8]{inputenc}
\usepackage{fancyhdr}
\pagestyle{fancy}

\setlength{\parindent}{0pt}
\setlength{\parskip}{1ex}

% Kopfzeile
\lhead{Optimierung B}
\chead{Übung 1}
\rhead{Niklas Fischer 298418 \\ Gereon Kremer 288911}

% Fußzeile
\lfoot{}
\cfoot{Seite \thepage{}}
\rfoot{}

\renewcommand{\thesection}{}
\renewcommand{\thesubsection}{(\alph{subsection})}

\begin{document}

\section{Aufgabe 1}

\subsection{}
Gesucht ist ein Kreis $K = v_0 e_1 v_1 ... e_n v_n$ im ungerichteten Graphen $G
= (V, E)$, so dass $E = \{ e_1, ..., e_n \}$, wobei das Gewicht
$g = min \sum\limits_{i=1}^{n} \delta(e_n)$ minimal ist.

\subsection{}
In einem eulerschen Graphen existiert ein Eulerkreis. Der Postbote muss dann
keinen Weg doppelt laufen und es gilt $g = \sum\limits_{e \in E} \delta(e)$.

\subsection{}
\begin{enumerate}
\item Sei $V_{odd} \subseteq V$ die Menge der Knoten mit ungeradem Grad.
\item Sei $E_{odd}$ eine Menge von Kanten mit $(v_1, v_2) \in E_{odd}$ für alle $v_1, v_2 \in V_{odd}$ und $\delta^*(v_1, v_2)$ die Länge des kürzesten Pfades zwischen $v_1$ und $v_2$.
\item Berechne ein (optimales, perfektes) Matching $M$ auf $G_{odd} = (V_{odd}, E_{odd})$.
\item Für jedes $m = (u_m, v_m) \in M$: Füge neuen Knoten $r_m$ in $V_{odd}$ und Kanten $(u_m, r_m), (r_m, v_m)$ in $E_{odd}$ mit $\delta(u_m, r_m) = \delta^*(u_m, v_m)$, $\delta(r_m, v_m) = 0$.
\item Berechne Eulertour auf dem so ergänzten Graphen.
\item Ersetze vorher ergänzte Kanten und Knoten durch den kürzesten Pfad zwischen den beiden Knoten.
\item Gebe den so erzeugten Kreis zurück.
\end{enumerate}

Beweis der Korrektheit:

In der optimalen Tour müssen alle Kanten mindestens einmal enthalten sein.
Sei $V_{odd}$ die Menge der Knoten mit ungeradem Grad.
Ist $V_{odd} = \emptyset$, so existiert eine Eulertour, die automatisch
optimal ist, da jede Kante nur einmal enthalten ist. Der Algorithmus liefert
in diesem Fall genau eine solche Tour zurück.

Falls $V_{odd} \neq \emptyset$ so ist $| V_{odd} |$ gerade. Da der erzeugte
Graph $(V_{odd}, E_{odd})$ vollständig ist, existiert stets ein optimales,
perfektes Matching. Es bleibt zu zeigen, dass die mithilfe des Matchings
konstruierte Eulertour optimal ist.

Da jeder Knoten in einem Kreis zu gerade vielen Kanten des Kreises
inzident ist, müssen bei jedem Knoten in $V_{odd}$ ungerade viele Kanten, also
insbesondere mindestens eine Kante, doppelt verwendet werden, um eine Eulertour
zu erhalten.

Wir betrachten nun die von den doppelt verwendeten Kanten induzierten Pfade.
Von jedem Knoten in $V_{odd}$ startet ein solcher Pfad $p$. Würde $p$ in
einem Knoten aus $V \setminus V_{odd}$ enden, so hätte dieser Knoten
dann ungeraden Grad, was der Anforderung widerspricht, am Ende eine
Eulertour im erweiterten Graphen zu berechnen. Daher muss der Pfad fortgeführt werden und ein einem
Knoten aus $V_{odd}$ enden. 

Es müssen also Pfade zwischen je zwei Knoten aus $V_{odd}$ existieren, deren
Kanten doppelt verwendet werden. Offensichtlich müssen diese Pfade möglichst
kurz gehalten werden. Dies wird durch die Berechnung eines optimalen
Matchings erreicht.

\section{Aufgabe 4}

Beweis:

Gegeben sei ein Graph $G = (V, E)$ mit $\deg(v) \geq \delta \; \forall v \in V$.
Laut Vorlesung ist $G$ ein einfacher Graph, er enthält also keine Schlingen
oder parallele Kanten.

Beginne an beliebigen Knoten $v_0 \in V$. Es existieren dann mindestens
$\delta$ Kanten von $v_0$ aus, insbesondere also eine Kante $e_0 = \{v_0,
v_1\}$.

Da $G$ einfach und $\deg(v_i) \geq \delta$ existiert von jedem Knoten $v_i$
aus mindestens eine Kante, die nicht zu einem Knoten in $\{
v_{\max(i-\delta+1, 0)}, ..., v_i\}$ führt. Führt diese Kante zu einem
bereits besuchten Knoten, so ist direkt ein Kreis der gewünschten Länge
gefunden.

Da $G$ endlich muss aber nach endlich vielen Schritten ein letzter Knoten $v_k$
gefunden werden und dessen Kanten allesamt zu bereits besuchten Knoten führen.
Da insbesondere auch $\deg(v_k) \geq \delta$, muss mindestens eine Kante zu
einem Knoten in $\{ v_0, ..., v_{k-\delta}\}$ führen und einen Kreis der
Länge mindestens $\delta + 1$ schließen.

Ist der letzte Knoten $v_k$ mit einer Kante $e_k = \{v_k, v_j\}$ mit $j \in
\{0, ..., k-\delta-1\}$, so ist der Pfad $v_j e_j ... e_{i-1} v_i e_i v_j$ ein Kreis der Länge $L > \delta$, insbesondere
also $L \geq \delta + 1$.

\end{document}
