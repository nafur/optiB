\documentclass[a4paper]{article}
\usepackage[utf8]{inputenc}
\usepackage{fancyhdr}
\usepackage{amsmath}
\usepackage[ngerman]{babel}
\usepackage{amsthm}
\usepackage{tikz}
\usepackage{listings}
%\usepackage{fullpage}

\lstset{numbers=left, basicstyle=\ttfamily, numberstyle=\tiny, mathescape=true} %listing style für code

\usetikzlibrary{positioning, trees, snakes}
\usetikzlibrary{automata, positioning, arrows, calc}

\setlength{\parindent}{0pt}
\setlength{\parskip}{1ex}
%\setlength{\headheight}{30pt}
\addtolength{\textwidth}{2in}
\addtolength{\textheight}{1.5in}
\addtolength{\hoffset}{-1in}
\addtolength{\voffset}{-0.75in}
\pagestyle{fancy}

% Kopfzeile
\lhead{Optimierung B}
\chead{Übung 6}
\rhead{Niklas Fischer 298418 \\ Gereon Kremer 288911}

% Fußzeile
\lfoot{}
\cfoot{Seite \thepage{}}
\rfoot{}

\renewcommand{\thesection}{}
\renewcommand{\thesubsection}{(\alph{subsection})}

\begin{document}

\section{Aufgabe 1}

\section{Aufgabe 4}
	Sei $G=(V,E)$ ein euklidischer Graph.
	Sei $C \subseteq E$ eine Lösung des TSPs.
	Angenommen es existieren zwei sich überschneidende Kanten $ab \in C$ (von $a \in V$ nach $b \in V$), und $cd \in C$ (von $c \in V$ nach $d \in V$). 
	
	Falls $a$, $b$, $c$, $d$ colinear sind, gilt die Aussage nicht:
	Sei $G$ ein volltändiger Graph mit $V = \{a,b,c,d\}$, und $a$, $b$, $c$, $d$ seien colinear. Dann ist der Zykel $C_1 = (ab,bc,cd,da)$ eine Lösung des TSP. Dies ist ein Widerspruch zu Behauptung.
	
	Falls $a$, $b$, $c$, $d$ collinear sind, colinear sind existiert $S$, der Schnittpunkt der beiden Kanten $ab$ und $cd$. Dann 	gilt:
	\begin{align*}
		|ab|+|cd| = (|aS| + |Sb|) + (|cS| + |Sd|) = (|aS| + |Sd|) + (|bS| + |Sc|) \stackrel*< |ad| + |bc|
	\end{align*}
	zu *: die Gleichheit würde genau dann gelten, wenn $a$, $S$ und $d$ sowie $b$, $S$ und $c$ colinear sind. Aufgrund der Konstruktion von $S$ würde das jedoch bedeuten, dass $a$, $b$, $c$, und $d$ kolinear sind. 
	
	Das Ersetzen von $ab$, und $cd$ im Zykel $C$ durch $ad$ und $cd$ und Umnummerieren von $C$ führt also zu einem kürzeren Zykel $C'$ der immer noch alle Knoten besucht. Dies ist ein Widerspruch dazu, dass $C$ Lösung des TSP ist. Also gilt die Behauptung.
\end{document}
